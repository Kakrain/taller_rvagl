% IEEE Paper Template for US-LETTER Page Size (V1)
% Sample Conference Paper using IEEE LaTeX style file for US-LETTER pagesize.
% Copyright (C) 2006-2008 Causal Productions Pty Ltd.
% Permission is granted to distribute and revise this file provided that
% this header remains intact.
%
% REVISION HISTORY
% 20080211 changed some space characters in the title-author block
%
\documentclass[10pt,conference,letterpaper]{IEEEtran}
\usepackage{times,amsmath,epsfig}
%
\title{Mesa PingTime y Simulaci\'on de un Paisaje}
%
\author{%
% author names are typeset in 11pt, which is the default size in the author block
{David Barrera{\small $~^{\#1}$}, Sergio Moncayo{\small $~^{*2}$}}%
% add some space between author names and affils
\vspace{1.6mm}\\
\fontsize{10}{10}\selectfont\itshape
% 20080211 CAUSAL PRODUCTIONS
% separate superscript on following line from affiliation using narrow space
$^{\#}$\,Facultad de Ingiene\'ia El\'ectrica y Computacional, Escuela Superior Polit\'ecnica del Litoral\\
Guayaquil, Ecuador\\
\fontsize{9}{9}\selectfont\ttfamily\upshape
%
% 20080211 CAUSAL PRODUCTIONS
% in the following email addresses, separate the superscript from the email address 
% using a narrow space \,
% the reason is that Acrobat Reader has an option to auto-detect urls and email
% addresses, and make them 'hot'.  Without a narrow space, the superscript is included
% in the email address and corrupts it.
% Also, removed ~ from pre-superscript since it does not seem to serve any purpose
$^{1}$\,author@espol.edu.ec\\
\fontsize{9}{9}\selectfont\ttfamily\upshape
% 20080211 CAUSAL PRODUCTIONS
% removed ~ from pre-superscript since it does not seem to serve any purpose
$^{2}$\,smoncayo@espol.edu.ec
}
%
\begin{document}
\maketitle
%
\begin{abstract} 
Como todos sabemos que en la \'ultima era, la tecnolog\'ia a crecido de manera sorprendente, cada vez nos interesa crear juegos que sean m\'as realista. Este campo se llamada la realidad aumentada; consiste en crear una realidad mixta en tiempo real, mezclando elementos reales y virtuales. Es decir, 
'a\~nades informaci\'on virtual sobre la realidad f\'isica, de modo que 
a trav\'es de una pantalla (m\'ovil, iPad, ordenador�) puedes visualizar 
una mezcla entre la realidad y el juego, puedes a\~nadir datos de 
inter\'es a lo que ves o incluso im\'agenes que complementen tu realidad. Por otra parte sabemos que las personas con cierta discapacidad no pueden viajar y realizar su sue\~no de viajar al su lugar deseado; lo que se trata hacer, con la ayuda de la realidad aumentada llevarla a un cuarto especializado y recreado al sitio donde el enfermos quieran ir. La mesa pingTime va dirigida a los jugadores de ping pong tradicional, debido que en ciertas ocasiones un jugador se aburre de lo rutinario que puede ser este deporte; pero con esta mesa que detecta el movimiento de la pelota y cambia de manera aleatorea su superficie para que no exista un patron y no sea rudimentario; este juego se hiciera mucho m\'as interesante de jugarlo.
\end{abstract}

% NOTE keywords are not used for conference papers so do not populate them
% \begin{keywords}
% keyword-1, keyword-2, keyword-3
% \end{keywords}
%
\section{Introducc\'ion}

Se debe de dar primero lo que es la realidad aumenta; es definir qu\'e realidad 
queremos aumentar, por ejemplo en uno de los ejemplos: un probador de ropa virtual que permite que los usuarios de esta aplicaci\'on pueden probarse la prenda que quieren de forma virtual sin tener que ir a la tienda f\'isicamente. La realidad o situaci\'on que se mejor\'o en ese ejemplo fue la de probarse ropa.
Con respecto a la mesa PingTime;fue creado Sergiu Doroftei, Bogdan Susma, Ion Cotenescu y Silviu Badea para el festival de arte y m\'usica electr\'onica Rokolectiv. La mesa Pingtime al final no solo es una especie de tablero de Ping Pong en esteroides, tambi\'en mide los tiempos de reacci\'on de los jugadores en cada partida con un rudimentario sistema de medici�n instalado en las raquetas.[https://www.prote.in/en/feed/2013/07/pingtime].
Con respecto a los paisajes, se los puede recrear de manera simultanea gracias al internet que puedes encontrar ciertos paisajes que puede ser del agrado del usuario y mezclarlos con sonidos y asi obtener una experiencia \'unica. [http://www.hispasonic.com/blogs/project-paperclip-fotografia-realidad-aumentada-paisajes-sonoros-interactivos/37169 ].
 

% the following command shrinks the final page to force the columns to
% be balanced.  You will need to adjust the value according to the 
% appearance of your last page.  Start by setting the value to 0mm
% and slowly increase it until the columns balance.  Alternatively,
% use balance.sty to do the job.
\enlargethispage{-62mm}

\section{Conclusi\'on}


\bibliographystyle{IEEEtran}

\end{document}
